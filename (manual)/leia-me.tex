\documentclass[a4paper,10pt]{article}
\usepackage[utf8]{inputenc}
\usepackage[brazil,brazilian]{babel}

%opening
\title{Manual para o tema de \LaTeX \\ Teses e dissertações - UFViçosa}
\author{Vinícius Barros Rodrigues\\\texttt{viniciusbrbio@gmail.com}}

\begin{document}

\maketitle

\section*{Como usar:}

\begin{enumerate}
 \item Inicialmente, preencha as informações dentro da pasta \texttt{preambulo}. Essa pasta contém os seguintes arquivos:
 \begin{description}
  \item \textbf{capa.tex}: preencher o seu nome completo, o título da tese, nome do programa, o título (Magister Scientiae ou Doctor Scientiae), cidade e data.
  \item \textbf{aprovacao.tex}: preencher o seu nome completo, o título da tese, nome do programa, o título (Magister Scientiae ou Doctor Scientiae), data de aprovação e membros da banca. O mês deve ser preenchido por extenso. O nome dos membros \textbf{não} deve conter prof. ou dr.. Caso o número de membros na sua banca for menor ou maior do que o padrão do arquivo, basta modificar as linhas.
  \item \textbf{dedicatória.tex};
  \item \textbf{epígrafe.tex};
  \item \textbf{agradecimentos.tex};
  \item \textbf{resumo.tex};
  \item \underline{não compile nenhum dos arquivos anteriores.}
  \end{description}
\item coloque todas as figuras e gráficos do seu trabalho na pasta \texttt{imagens};
\item mova o seu arquivo de referências (.bib) para o diretório da tese. De preferência, renomeie o seu arquivo para ``referencias.bib'' e substitua o arquivo original.
\item a pasta \texttt{capitulos} contém os arquivos para o preenchimento com os textos:
  \begin{description}
    \item \textbf{introducao.tex}: apague o arquivo de exemplo e cole o seu texto. Cuidado para não apagar o cabeçalho original. No cabeçalho, preencha com o seu nome e com o título resumido da tese nos locais indicados. Caso deseje, pode deixar em branco essas opções.
    \item \textbf{cap02.tex}: apague o exemplo e cole o seu texto. Não precisa preencher cabeçalho.
    \item \underline{se necessário, crie outros capítulos.}
   \end{description}
\item abra o arquivo \texttt{main.tex}:
\begin{itemize}
 \item \textit{verifique se todos os pacotes necessários estão instalados no seu sistema};
\end{itemize}
\item verifique se os capítulos estão adicionados com o comando \texttt{imput}. Adicione ou remova-os de acordo com sua necessidade;
\item confirme o nome do seu arquivo de referências \texttt{.bib};
\item caso necessário, você pode adicionar um arquivo \texttt{.pdf} com análises estatísticas no fim do arquivo, com o comando \texttt{includepdf}.
\item \underline{compile}. \textbf{Esse é o único arquivo que pode ser compilado}.
\end{enumerate}



\end{document}
